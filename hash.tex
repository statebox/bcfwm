\subsection{Secure Hashing}

The most basic block-chain functionally is provided by cryptographically hash functions.

\paragraph{Hash Function}

A \emph{hash function} $H$ is any function that can map arbitrary large bit-strings to a fixed size ($m$) bit string.

\[
	H : \bits{\star} \to \bits{m}
\]

The result of applying $H$ to some value $x$ is called the \emph{digest} or \emph{the hash value} (of $x$).

\paragraph{Cryptographically Secure Hash Function}

A \emph{one way function} is a functions $f$ whose inverse $f^{-1}$ or pre-image $\{x | f(x) = y \}$ cannot be \emph{efficiently computed}.

The existence of one way functions is still an open problem, a proof of which would imply that $P \neq NP$.

Assuming they exist and given just the output $y$, the only way to find a $x$ 
such that $f(x)=y$ is by brute force: try all possible values of $x$ until a
match is found. This is exploited in Proof of Work, which we talk about
later, see \ref{PoW}.

A \emph{cryptographically secure hash function} is a hash function that is also a \emph{one way function}.

The output of such hash functions must be indistinguishable from a (pseudo-)random function and the domain should distribute evenly over the co-domain: flipping just a single bit of an input string $p$ to yield $p'$ should have $f(p)$ and $f(p')$ return radically different digest values.

If $p \neq p'$ but $f(p) = f(p')$, we speak of a \emph{hash collision}, this is a \emph{Very Bad Thing}. A hash function is called \emph{collision free} if any efficient algorithm has negligible probability of finding such a collision.

The \texttt{SHA} family of hash functions is very well know, \texttt{SHA256} is a commonly used cryptographic hash function, considered to be cryptographically secure and collision free. 
