\subsection{Asymmetric Cryptography}
% TODO comment on reuse of X and rainbow tables

Cryptography is used to sign and verify transactions, but it can also be used to encrypt and decrypt information.
Asymmetric refers to the fact the two operations use different, but related keys, called the \emph{secret} and \emph{public} keys.

Two well know cryptographic systems are \emph{RSA} (Rivest-Shamir-Adleman) and \emph{ECC} (Elliptic Curve Cryptography), however, their details are irrelevant for our purpose.

Let $\alpha$ be some ``actor'' in our system.

We then $SK_\alpha$ for his or her or it's \emph{secret key}, which you can think of as a very large random number (actually a element of a large finite field).

There is a procedure to derive from $SK_\alpha$ a corresponding \emph{public key}, $PK_\alpha$.

The two keys together are called a keypair and written as $\left< SK_\alpha, PK_\alpha \right>$.

We define the actor-address $\tilde{\alpha} := H(PK_\alpha)$.

We can think of the system as having four operations:

\begin{itemize}
\item \texttt{encrypt}: $\star \times SK_{\alpha} $
\end{itemize}
